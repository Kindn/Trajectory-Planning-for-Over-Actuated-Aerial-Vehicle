% !TEX root = ../main.tex

% 附录2
\chapter{基于欧拉角姿态表示的罚函数梯度}\label{appdx:B}

从附录\ref{appdx:A}中给出的罚函数梯度表达形式可以发现,
不同的姿态表示法对应的罚函数梯度表达式之间的差别仅体现在角速度和旋转矩阵部分,
因此仅需对角速度和旋转矩阵的梯度进行推导,再代入附录\ref{appdx:A}中的式子即可。

根据\equref{equ:euler_angle_to_omega},
可以得到$\bm{\omega}$关于系数矩阵的梯度为: 
\begin{align}
    & \pdiff{\omega_x}{\bm{c}_i^{\sigma}} = 
    \pdiff{\dot{\phi}}{\bm{c}_i^{\sigma}}\cos\theta - 
    \dot{\phi}\pdiff{\theta}{\bm{c}_i^{\sigma}}\sin\theta - 
    \pdiff{\dot{\psi}}{\bm{c}_i^{\sigma}}\cos\phi\sin\theta + 
    \dot{\psi}\pdiff{\phi}{\bm{c}_i^{\sigma}}\sin\phi\sin\theta - 
    \dot{\psi}\pdiff{\theta}{\bm{c}_i^{\sigma}}\cos\phi\cos\theta \label{equ:omegax_d_cis} \\
    & \pdiff{\omega_y}{\bm{c}_i^{\sigma}} = 
    \pdiff{\dot{\theta}}{\bm{c}_i^{\sigma}} + 
    \pdiff{\dot{\psi}}{\bm{c}_i^{\sigma}}\sin\phi + 
    \dot{\psi}\pdiff{\phi}{\bm{c}_i^{\sigma}}\cos\phi \label{equ:omegay_d_cis} \\
    & \pdiff{\omega_z}{\bm{c}_i^{\sigma}} = 
    \pdiff{\dot{\phi}}{\bm{c}_i^{\sigma}}\sin\theta + 
    \dot{\phi}\pdiff{\theta}{\bm{c}_i^{\sigma}}\cos\theta + 
    \pdiff{\dot{\psi}}{\bm{c}_i^{\sigma}}\cos\phi\cos\theta - 
    \dot{\psi}\pdiff{\phi}{\bm{c}_i^{\sigma}} - 
    \dot{\psi}\pdiff{\theta}{\bm{c}_i^{\sigma}}\cos\phi\sin\theta \label{equ:omegaz_d_cis} \\
    & \pdiff{\omega_x}{\bm{c}_i^{p}} = 
    \pdiff{\omega_y}{\bm{c}_i^{\sigma}} = 
    \pdiff{\omega_z}{\bm{c}_i^{\sigma}} = \textbf{0} \label{equ:omegaxyz_d_cip}
\end{align}
$\bm{\omega}$关于时间分配向量的梯度以及$\bm{R}$的相关梯度可以同理得到。
其中要求${\partial \bm{R}}/{\partial [\bm{c}_i^{\sigma}]_{mn}}$可以先求$\bm{R}$中的 
每个元素$r_{ij},i,j\in\{1,2,3\}$关于$\bm{c}_i^{\sigma}$的梯度,然后则有:
\begin{equation}
    \pdiff{\bm{R}}{[\bm{c}_i^{\sigma}]_{mn}} = 
    \left[\left[\pdiff{r_{ij}}{\bm{c}_i^{\sigma}}\right]_{mn}\right]_{i=1,j=1}^{3,3}
\end{equation}

