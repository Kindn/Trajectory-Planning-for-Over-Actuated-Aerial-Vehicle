% !TEX root = ../main.tex

% 结论
\begin{conclusions} 

本文重点研究冗余驱动多旋翼飞行器在复杂环境中的$SE(3)$轨迹规划。
通过深入研究、对比各种已有的轨迹规划方案,结合冗余驱动旋翼飞行器的不同性质,完成了规划算法设计、代码实现、仿真验证及实物验证。
根据研究结果,总结出结论如下:
\begin{enumerate}
    \renewcommand{\labelenumi}{(\theenumi)}
    \item 对OmniHex的结构设计进行了介绍,基于实际系统建立并分析了OmniHex的动力学模型,
    阐明了其是微分平坦系统并给出了平坦输出的形式,
    为后续选取平坦输出进行6自由度$SE(3)$轨迹规划提供了依据。
    \item 对$SE(3)$空间中的RRT算法进行设计和实现,
    确定了$SE(3)$状态空间中的采样、插值和度量等操作;
    实现了GNAT数据结构;
    设计并实现了基于点云地图和基于八叉树地图的两种有效性检测方法,
    并对设计好的RRT算法进行了效果测试及分析;
    随后设计并实现了安全飞行走廊的生成算法,并给出了实测效果。
    为后端部分提供了可行路径和安全约束等初始信息。
    \item 根据冗余驱动旋翼飞行器的特点,结合几何约束优化的核心思想,
    建立了适用于冗余驱动旋翼飞行器6自由度$SE(3)$轨迹优化的问题形式,
    并给出了相关公式的推导,设计出了几何约束下冗余驱动旋翼飞行器轨迹优化的算法框架;
    并给出了基于欧拉角和基于四元数的两种适用于本框架的姿态参数化方式。
    \item 通过实验对比得出了基于欧拉角和基于四元数的两种规划方式各自的优势与短板:
    欧拉角法的主要优势为计算速度通常较快,且规划出的轨迹较为平滑,
    缺点是对障碍物不够敏感,规划出的轨迹经常发生擦碰,
    且在极端环境中有一定几率会触发严重碰撞;
    四元数法的主要优势为对障碍物敏感,
    规划出的轨迹安全性较高,
    其缺点除了平滑度稍差外,就是计算时间稍长。
    \item 通过仿真与实际飞行,初步验证了本课题设计的规划算法具有可行性。
    针对于冗余驱动旋翼飞行器6自由度避障轨迹规划的相关研究极少,
本文尝试为这个领域探索出了一个可行的方向。
\end{enumerate}

由于研究时间限制,本课题的研究还存在着很大的提升空间。
比如后端优化的计算效率远未达到针对于四旋翼飞行器的官方实现的水平;
未考虑实际的飞行器在狭小空间内飞行时的空气动力学效应。
本课题后续研究计划与展望如下:
\begin{enumerate}
    \renewcommand{\labelenumi}{(\theenumi)}
    \item 对现有代码进行优化,并考虑实现使用并行计算来加速后端优化中的梯度计算,尽可能提高规划器的计算效率,达到能够进行实时规划的水平;
    \item 为OmniHex加上机载感知系统,如视觉惯性里程计等,配合规划算法实现其自主建图、定位与规划。
\end{enumerate}

\end{conclusions}
