% !TEX root = ../main.tex

\hitszsetup{
  %******************************
  % 注意:
  %   1. 配置里面不要出现空行
  %   2. 不需要的配置信息可以删除
  %******************************
  %
  %=====
  % 秘级
  %=====
  statesecrets={公开},
  natclassifiedindex={TM301.2},
  intclassifiedindex={62-5},
  %
  %=========
  % 中文信息
  %=========
  ctitleone={过驱动飞行器的轨迹},%本科生封面使用
  ctitletwo={规划},%本科生封面使用
  ctitlecover={过驱动飞行器的轨迹规划},%放在封面中使用,自由断行
  ctitle={过驱动飞行器的轨迹规划},%放在原创性声明中使用
  csubtitle={一条副标题}, %一般情况没有,可以注释掉
  cxueke={工学},
  cpostgraduatetype={学术},
  csubject={自动化},
  % csubject={机械工程},
  caffil={机电工程与自动化学院},
  % caffil={哈尔滨工业大学(深圳)},
  cauthor={刘培焱},
  csupervisor={陈浩耀 教授},
  cassosupervisor={某某某 教授}, % 副指导老师
  % ccosupervisor={某某某 教授}, % 联合指导老师
  % 日期自动使用当前时间,若需指定按如下方式修改:
  cdate={2022年6月},
  % 指定第二页封面的日期,即答辩日期
  cdatesecond={2022年06月10日},
  cstudentid={180310211},
  cstudenttype={同等学力人员}, %非全日制教育申请学位者
  %(同等学力人员)、(工程硕士)、(工商管理硕士)、
  %(高级管理人员工商管理硕士)、(公共管理硕士)、(中职教师)、(高校教师)等
  %
  %
  %=========
  % 英文信息
  %=========
  etitle={Research on robot intelligent grasping based on Neural Network},
  esubtitle={This is the sub title},
  exueke={Engineering},
  esubject={Mechanical Engineering},
  eaffil={Harbin Institute of Technology, Shenzhen},
  eauthor={Jingxuan Yang},
  esupervisor={Prof. XXX},
  % eassosupervisor={XXX},
  % 日期自动生成,若需指定按如下方式修改:
  edate={June, 2020},
  estudenttype={Master of Engineering},
  %
  % 关键词用“英文逗号”分割
  ckeywords={过驱动飞行器, 轨迹规划, 避障,SE(3)规划},
  ekeywords={over-actuated aircraft, trajectory planning, obstacle avoidance, SE(3) planning},
}

% 中文摘要
\begin{cabstract}
  进入21世纪以来,多旋翼无人机领域取得了很大的发展,成为一类成功从实验室走进人们生活的机器人系统。
  为突破现在市场上主流的欠驱动无人机所存在的瓶颈,近年来人们又研制出了不少种类过驱动多旋翼无人机系统。
  作为一种全驱动系统,这种无人机可以跟踪6自由度轨迹,有效增强了多旋翼无人机的机动性能,拓宽了其应用场景。
  可以想到,这会让多旋翼飞行器在复杂、拥挤的环境中的自主飞行能力大幅提升,而有效的规划算法是实现这种提升的关键。
  现有的规划方案主要是基于传统欠驱动飞行器开发的,在过驱动飞行器上鲜有先例。

  本文以实现过驱动多旋翼飞行器在复杂环境中的避障规划为目标,为过驱动多旋翼飞行器设计了一个基于优化的全自由度全局轨迹规划器。
  该规划器根据由一系列凸多面体表示的安全约束,结合飞行器的整体形态以及动力学约束,最终规划出一条由起点到终点的无碰撞最优轨迹。

  本文理论部分介绍了规划器前端路径搜索算法以及安全约束生成算法的设计和实现,并且详细讲述了后端轨迹优化算法的原理和设计。
  其中在后端优化算法部分分别介绍了基于欧拉角和基于四元数的两种姿态规划方案,并在附录中给出了相应公式的推导。

  在实验部分中,本文以一种全向六旋翼飞行器机构为平台,针对不同的仿真场景进行了规划实验,以验证和对比基于上述两种姿态规划方案所设计规划器的可行性、有效性和计算效率。
  随后基于Gazebo和PX4进行了仿真实验,利用本文设计的规划器成功实现了全向六旋翼飞行器在狭长通道等极端环境中的避障飞行。
  最后在实物平台上进行实验,验证了规划器的实用性。



\end{cabstract}

% 英文摘要
\begin{eabstract}
  Since entering the 21st century, the field of multi-rotor UAV has achieved unprecedented development and become a kind of robot system that has successfully entered people's life from the laboratory.
In order to break through the bottleneck of underactuated UAV in the current market, many kinds of over-actuated multi-rotor UAV systems have been developed in recent years.
As a fully actuated system, the UAV can track a 6-DoF trajectory, effectively enhancing the maneuverability of the multi-rotor UAV and broadening its application scenarios.
As you can imagine, this would greatly improve the autonomous flight capability of multi-rotor aircraft in complex, crowded environments, and an effective planning algorithm is the key to achieving this improvement.
The existing planning schemes are mainly based on traditional underactuated aircraft, and there are few precedents for overactuated aircraft.

In order to realize obstacle avoidance planning of over-actuated multi-rotor aircraft in complex environment, a global trajectory planner with full degree of freedom based on optimization is designed for over-actuated multi-rotor aircraft.
According to the safety constraints represented by a series of convex polyhedra, combined with the whole-body shape and dynamics constraints of the aircraft, the planner finally plans an optimal collision free trajectory from the starting point to the end point.

The theoretical part of this thesis introduces the design and implementation of the front-end path search algorithm and the security constraint generation algorithm of the planner, and describes the principle and design of the back-end trajectory optimization algorithm in detail.
In the part of the back-end optimization algorithm, two attitude planning schemes based on Euler Angle and quaternion are introduced respectively, and the derivation of corresponding formulas is given in the appendix.

In the experimental part, planning experiments are carried out for different simulation scenarios with an omnidirectional hexarotor aircraft mechanism as the platform to verify and compare the feasibility, effectiveness and computational efficiency of the planners designed based on the above two attitude planning schemes.
Then, simulation experiments are carried out based on Gazebo and PX4, and the obstacle avoidance flight of omnidirectional hexarotor aircraft in narrow and long passage and other extreme environments is successfully realized by using the planner designed in this thesis.
Finally, experiments are carried out on a physical platform to verify the practicality of the planner.
\end{eabstract}
