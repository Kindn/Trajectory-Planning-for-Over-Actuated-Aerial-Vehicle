\documentclass[11pt]{ctexart}  
\usepackage[top=2cm, bottom=2cm, left=2cm, right=2cm]{geometry}  
\usepackage{algorithm}  
\usepackage{algorithmicx}  
\usepackage{algpseudocode}  
\usepackage{amsmath}  
  
\floatname{algorithm}{算法}  
\renewcommand{\algorithmicrequire}{\textbf{输入:}}  
\renewcommand{\algorithmicensure}{\textbf{输出:}}  
  
\begin{document}  
    \begin{algorithm}[H]
		\renewcommand{\thealgorithm}{2-1}
        \caption{生成均匀分布的随机单位四元数}  
        \begin{algorithmic}[1] %每行显示行号  
            \Require无  
            \Ensure 均匀随机四元数$\mathbf{q}=(w,x,y,z)$
            \STATE $s=uniform(0,1)$ // uniform(0,1)为生成0到1之间的均匀分布随机数
            \STATE $\sigma_{1}=\sqrt{1-s}$
			\STATE $\sigma_{2}=\sqrt{s}$
			\STATE $\theta_{1}=2\pi*uniform(0,1)$
			\STATE $\theta_{2}=2\pi*uniform(0,1)$
			\STATE $w=\cos(\theta_{2})*\sigma_{2}$
			\STATE $x=\sin(\theta_{1})*\sigma_{1}$
			\STATE $y=\cos(\theta_{1})*\sigma_{1}$
			\STATE $z=\sin(\theta_{2})*\sigma_{2}$
			\STATE \Return{$(w,x,y,z)$}
        \end{algorithmic}  
    \end{algorithm} 

	  \begin{algorithm}[H]
		\renewcommand{\thealgorithm}{2-2}
        \caption{生成无碰撞椭球}  
        \begin{algorithmic}[1] %每行显示行号  
            \Require 线段$L_{i}= \langle \mathbf{p}_{i} \rightarrow \mathbf{p}_{i+1} \rangle $,障碍物点集$O$
            \Ensure 无碰撞椭球$\xi$
            \STATE 初始化$\xi$为以$L_{i}$为直径的球体,且以$\mathbf{p}_{i}\mathbf{p}_{i+1}$为$\xi$的$\widetilde{x}$轴
			\STATE  $O_{inside}\gets \xi.getInside(O)$
			\While {$O_{inside} \neq \emptyset$}
				\STATE $\mathbf{p}\gets \min_{\mathbf{q} \in O_{inside}}\Vert \mathbf{E}^{-1}(\mathbf{p}-\mathbf{d}) \Vert$
				\STATE 保持$\widetilde{x}$轴不变,$b=c$,调整$\xi$使其的边界经过点$\mathbf{p}$
				\STATE $O_{inside}\gets \xi.getInside(O)$
			\EndWhile
			\STATE 此时$\xi$接触一障碍物点$\mathbf{p}^{*}$,将其与$\xi$的$\widetilde{x}$轴确定的平面定为$\xi$的$\widetilde{x}-\widetilde{y}$平面
			\STATE 沿$\widetilde{z}$轴调整$\xi$使$c=a$
			\STATE  $O_{inside}\gets \xi.getInside(O)$
			\While {$O_{inside} \neq \emptyset$}
				\STATE $\mathbf{p}\gets \min_{\mathbf{q} \in O_{inside}}\Vert \mathbf{E}^{-1}(\mathbf{p}-\mathbf{d}) \Vert$
				\STATE 沿$\widetilde{z}$轴调整$\xi$使其的边界经过点$\mathbf{p}$
				\STATE $O_{inside}\gets \xi.getInside(O)$
			\EndWhile
			\STATE \Return $\xi$
        \end{algorithmic}  
    \end{algorithm} 

	\begin{algorithm}[H]
		\renewcommand{\thealgorithm}{2-3}
        \caption{给定无碰撞椭球$\xi^{0}(\mathbf{E},\mathbf{d})$和障碍物点集$O$,求无碰撞凸多面体$C(\mathbf{A},\mathbf{b})$}  
        \begin{algorithmic}[1] %每行显示行号  
            \Require 无碰撞椭球$\xi^{0}(\mathbf{E},\mathbf{d})$,障碍物点集$O$ 
            \Ensure 凸多面体$C(\mathbf{A},\mathbf{b})$
            \STATE  $O_{remain}\gets \xi.getInside(O),j \gets 0$
			\While {$O_{remain} \neq \emptyset$}
				\STATE $\mathbf{p}_{j}^{c} \gets \min_{\mathbf{q} \in O_{inside}}\Vert \mathbf{E}^{-1}(\mathbf{p}-\mathbf{d}) \Vert$ 
				\STATE 按比例膨胀$\xi^{0}$使其表面经过$\mathbf{p}_{j}^{c}$
				\STATE $\mathbf{a}_{j} \gets 2\mathbf{E}^{-1}\mathbf{E}^{-\text{T}}(\mathbf{p}_{j}^{c}-\mathbf{d})$
				\STATE $b_{j} \gets \mathbf{a}_{j}^{\text{T}}\mathbf{p}_{j}^{c}$
				\STATE 去除$O_{remain}$中不在半空间$H_{j}(\mathbf{a}_{j},b_{j})$之内的所有点
				\STATE $j \gets j+1$
			\EndWhile
			\STATE $\mathbf{A} \gets [\mathbf{a}_{0}\ \mathbf{a}_{1}\ \cdots]^{\text{T}}$,$\mathbf{b} \gets [b_0 \ b_1 \ \cdots]^{\text{T}}$
			\STATE \Return $C(\mathbf{A},\mathbf{b})$
        \end{algorithmic}  
    \end{algorithm} 

	
	
\end{document}  
