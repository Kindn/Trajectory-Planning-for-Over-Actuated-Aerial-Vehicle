% !TEX root = ../main.tex

\hitszsetup{
  %******************************
  % 注意:
  %   1. 配置里面不要出现空行
  %   2. 不需要的配置信息可以删除
  %******************************
  %
  %=====
  % 秘级
  %=====
  statesecrets={公开},
  natclassifiedindex={TM301.2},
  intclassifiedindex={62-5},
  %
  %=========
  % 中文信息
  %=========
  ctitleone={过驱动飞行器的轨迹},%本科生封面使用
  ctitletwo={规划},%本科生封面使用
  ctitlecover={过驱动飞行器的轨迹规划},%放在封面中使用,自由断行
  ctitle={过驱动飞行器的轨迹规划},%放在原创性声明中使用
  csubtitle={一条副标题}, %一般情况没有,可以注释掉
  cxueke={工学},
  cpostgraduatetype={学术},
  csubject={自动化},
  % csubject={机械工程},
  caffil={机电工程与自动化学院},
  % caffil={哈尔滨工业大学(深圳)},
  cauthor={刘培焱},
  csupervisor={陈浩耀 教授},
  cassosupervisor={某某某 教授}, % 副指导老师
  % ccosupervisor={某某某 教授}, % 联合指导老师
  % 日期自动使用当前时间,若需指定按如下方式修改:
  cdate={2022年6月},
  % 指定第二页封面的日期,即答辩日期
  cdatesecond={2022年06月10日},
  cstudentid={180310211},
  cstudenttype={同等学力人员}, %非全日制教育申请学位者
  %(同等学力人员)、(工程硕士)、(工商管理硕士)、
  %(高级管理人员工商管理硕士)、(公共管理硕士)、(中职教师)、(高校教师)等
  %
  %
  %=========
  % 英文信息
  %=========
  etitle={Research on robot intelligent grasping based on Neural Network},
  esubtitle={This is the sub title},
  exueke={Engineering},
  esubject={Mechanical Engineering},
  eaffil={Harbin Institute of Technology, Shenzhen},
  eauthor={Jingxuan Yang},
  esupervisor={Prof. XXX},
  % eassosupervisor={XXX},
  % 日期自动生成,若需指定按如下方式修改:
  edate={June, 2020},
  estudenttype={Master of Engineering},
  %
  % 关键词用“英文逗号”分割
  ckeywords={过驱动飞行器, 轨迹规划, 避障},
  ekeywords={over-actuated aircraft, trajectory planning, obstacle avoidance},
}

% 中文摘要
\begin{cabstract}

  全向多旋翼无人飞行器是近年来正在发展的一个研究领域。作为一种全向系统,
        全向多旋翼飞行器具有传统欠驱动多旋翼飞行器相比更加灵活的机动性,其在狭小
        空间中的避障飞行也具有更大的优势。本文中我们为全向六旋翼飞行器设计了一个
        基于优化的六自由度轨迹规划器,本规划器在欠驱动四旋翼飞行机的SE(3)运动规
        划方案的基础上根据全向六旋翼飞行器的特点进行改进,用一系列凸多面体表示
        环境中的安全区域,得到一条包含位置坐标和表示机体姿态的六自由度
        无碰撞轨迹。并进行了仿真实验验证了可行性。

\end{cabstract}

% 英文摘要
\begin{eabstract}
  Omnidirectional multirotor uav is a developing research 
  field in recent years. As an omnidirectional system, 
  omnidirectional multi-rotor aircraft has more flexible 
  maneuverability than traditional underactuated multi-rotor 
  aircraft, and it also has greater advantages in obstacle 
  avoidance flight in narrow space. In this paper, 
  we design a six-degree of freedom trajectory planner 
  based on optimization for omnidirectional six-rotor aircraft.
   Based on the SE(3) motion planning scheme of underactuated 
   four-rotor aircraft, the planner is improved according to the
    characteristics of omnidirectional six-rotor aircraft. 
    A series of convex polyhedra are used to represent the
     safety zone in the environment. A 6 dof collision-free 
     trajectory containing position coordinates and body attitude 
     is obtained. The feasibility is verified by simulation 
     experiments.

\end{eabstract}
